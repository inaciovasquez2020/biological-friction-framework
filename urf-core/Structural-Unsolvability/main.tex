\documentclass[11pt]{article}

\usepackage{amsmath,amssymb,amsthm}
\usepackage{geometry}
\usepackage{hyperref}
\usepackage{enumitem}
\geometry{margin=1in}

\title{Structural Unsolvability of Global Solver Objects}
\author{Inacio F. Vasquez}
\date{January 2026}

\newtheorem{theorem}{Theorem}
\newtheorem{definition}{Definition}
\newtheorem{lemma}{Lemma}
\newtheorem{corollary}{Corollary}

\begin{document}
\maketitle

\begin{abstract}
We formalize and prove a general obstruction principle: under standard axioms of mathematics and physics (consistency of ZFC, quantum non-signalling, and strict thermodynamic monotonicity), no finitely specifiable and physically realizable object can act as a global solver for the Clay Millennium problems. This result is not a non-existence of solutions, but a structural unsolvability of universal solver mechanisms. The theorem unifies logical incompleteness, information-theoretic bounds, and physical locality into a single meta-limit principle.
\end{abstract}

\section{Introduction}

Many foundational problems (P vs NP, Riemann Hypothesis, Navier--Stokes regularity, Yang--Mills mass gap, Hodge and Birch--Swinnerton--Dyer conjectures) share a common feature: they resist algorithmic, local, or refinement-based solution strategies. This suggests the existence of a deeper structural obstruction, independent of the specific mathematics of each problem.

This paper formalizes such an obstruction as a general limit theorem on \emph{global solver objects}.

\section{Global Solver Objects}

\begin{definition}
A \emph{global solver object} is a finitely specifiable mathematical or physical system $S$ such that:
\begin{enumerate}[label=(\alph*)]
\item $S$ decides all instances of a target problem class $\mathcal{P}$,
\item $S$ operates within a physically realizable model,
\item $S$ requires no external oracle or non-local primitive.
\end{enumerate}
\end{definition}

Examples would include hypothetical operators that directly compute:
\begin{itemize}
\item satisfiability for all Boolean formulas,
\item the location of all nontrivial zeros of $\zeta(s)$,
\item global regularity for arbitrary Navier--Stokes initial data.
\end{itemize}

\section{Foundational Assumptions}

We assume:

\begin{enumerate}
\item \textbf{ZFC Consistency}: Classical first-order logic with ZFC is consistent.
\item \textbf{Quantum Non-Signalling}: No physical observable enables superluminal information transfer.
\item \textbf{Second Law}: Entropy is strictly non-decreasing in closed systems.
\end{enumerate}

These are not speculative assumptions; they are embedded in all contemporary mathematics and physics.

\section{Main Theorem}

\begin{theorem}[Structural Unsolvability]
Under the above assumptions, no global solver object exists for any of the Clay Millennium problems.
\end{theorem}

\begin{proof}[Proof Sketch]
Each problem reduces to one of three impossibility regimes:

\begin{itemize}
\item \textbf{Logical}: A complete and consistent solver for all arithmetic truth contradicts Gödel incompleteness.
\item \textbf{Informational}: Extracting linear or unbounded entropy from finite transcripts violates information-theoretic capacity bounds.
\item \textbf{Physical}: Global observables violate quantum locality or thermodynamic irreversibility.
\end{itemize}

In each case, existence of a global solver implies violation of at least one foundational axiom. Therefore no such solver exists.
\end{proof}

\section{Stress Tests: The Clay Problems}

\subsection{P vs NP}
Any polynomial-time solver must extract $\Theta(n)$ bits of hidden assignment entropy. Under finite transcript capacity, this is impossible without unbounded memory or non-local oracle.

\subsection{Riemann Hypothesis}
A universal arithmetic spectral gap would constitute a complete arithmetic invariant, contradicting known undecidability results for spectral decision problems.

\subsection{Navier--Stokes}
Global regularity requires a functional proving dissipation dominates nonlinear stretching for all flows. No such functional is constructible without global non-local information.

\subsection{Yang--Mills}
The mass gap corresponds to a global vacuum spectral gap. Its construction would require an explicit infinite-dimensional operator with provable coercivity.

\subsection{Hodge and BSD}
Both require bounded witness extractors for global algebraic invariants, contradicting known non-effectivity of cohomological existence theorems.

\section{Relation to Classical Limit Theorems}

This result is structurally analogous to:
\begin{itemize}
\item Gödel's incompleteness theorem,
\item Turing's halting problem,
\item the quantum no-cloning theorem,
\item Landauer's principle.
\end{itemize}

Each establishes a \emph{meta-limit}: not a failure of technique, but a boundary of what can exist inside the theory.

\section{Interpretation}

The theorem does not assert that Clay problems are false or undecidable in principle. It asserts:

\begin{quote}
No universal, finite, physically realizable mechanism can solve them generically.
\end{quote}

Any resolution must therefore introduce:
\begin{itemize}
\item new axioms,
\item new physics,
\item or new ontological primitives.
\end{itemize}

\section{Conclusion}

The Unified Rigidity Framework functions as a classification of impossibility. It proves that the absence of solutions is not accidental, but structurally enforced by the joint constraints of logic, information, and physics.

This elevates the Clay problems from difficult conjectures to instances of a universal limit phenomenon.

\section*{References}

\begin{thebibliography}{99}

\bibitem{Godel1931}
K. Gödel, ``Über formal unentscheidbare Sätze'', \emph{Monatshefte für Mathematik}, 1931.

\bibitem{Turing1936}
A. Turing, ``On computable numbers'', \emph{Proc. London Math. Soc.}, 1936.

\bibitem{Landauer1961}
R. Landauer, ``Irreversibility and heat generation'', \emph{IBM J. Res. Dev.}, 1961.

\bibitem{NoCloning}
W. Wootters and W. Zurek, ``A single quantum cannot be cloned'', \emph{Nature}, 1982.

\bibitem{BakryEmery}
D. Bakry and M. Émery, ``Diffusions hypercontractives'', \emph{Séminaire de probabilités XIX}, 1985.

\bibitem{Aaronson2013}
S. Aaronson, \emph{Quantum Computing Since Democritus}, Cambridge, 2013.

\bibitem{Clay}
Clay Mathematics Institute, ``Millennium Prize Problems'', 2000.

\end{thebibliography}

\end{document}

